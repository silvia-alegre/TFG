\documentclass[11pt]{article}

\usepackage{ods_etsinf}

\begin{document}

\phantom{x}

\vspace{1ex}

\section*{ANEXO}

\vspace{2ex}

OBJETIVOS DE DESARROLLO SOSTENIBLE

\vspace{4ex}

Grado de relaci\'on del trabajo con los Objetivos de Desarrollo Sostenible (ODS).

\vspace{2ex}

\begin{tabular}{|l|c|c|c|c|}\hline
\textbf{Objetivos de Desarrollo Sostenible} & \textbf{Alto} & \textbf{Medio} & \textbf{Bajo} & \textbf{No} \\
& & & & \textbf{procede} \\ \hline
ODS 1.  \textbf{Fin de la pobreza.}                            & & & \textbf{X} &  \\ \hline
ODS 2.  \textbf{Hambre cero.}                                  & & & & \textbf{X} \\ \hline
ODS 3.  \textbf{Salud y bienestar.}                            & & \textbf{X} &  & \\ \hline
ODS 4.  \textbf{Educaci\'on de calidad.}                       & & \textbf{X} & & \\ \hline
ODS 5.  \textbf{Igualdad de g\'enero.}                         & & & & \textbf{X} \\ \hline
ODS 6.  \textbf{Agua limpia y saneamiento.}                    & & & & \textbf{X} \\ \hline
ODS 7.  \textbf{Energ\'{\i}a asequible y no contaminante.}     & & & & \textbf{X} \\ \hline
ODS 8.  \textbf{Trabajo decente y crecimiento econ\'omico.}    & & \textbf{X} & & \\ \hline
ODS 9.  \textbf{Industria, innovaci\'on e infraestructuras.}   & & & & \textbf{X} \\ \hline
ODS 10. \textbf{Reducci\'on de las desigualdades.}             & \textbf{X} & & & \\ \hline
ODS 11. \textbf{Ciudades y comunidades sostenibles.}           & & & & \textbf{X} \\ \hline
ODS 12. \textbf{Producci\'on y consumo responsables.}          & & & \textbf{X} & \\ \hline
ODS 13. \textbf{Acci\'on por el clima.}                        & & & & \textbf{X} \\ \hline
ODS 14. \textbf{Vida submarina.}                               & & & & \textbf{X} \\ \hline
ODS 15. \textbf{Vida de ecosistemas terrestres.}               & & & & \textbf{X} \\ \hline
ODS 16. \textbf{Paz, justicia e instituciones s\'olidas.}      & & & \textbf{X} & \\ \hline
ODS 17. \textbf{Alianzas para lograr objetivos.}               & & &  & \textbf{X} \\ \hline
\end{tabular}

\newpage

\phantom{x}

\vspace{1ex}

\textbf{Reflexi\'on sobre la relaci\'on del TFG/TFM con los ODS y con el/los ODS m\'as relacionados.}

\vspace{1ex}

La aplicación de modelos de lenguaje grandes en sistemas de comunicación aumentativa y alternativa tiene un impacto significativo, especialmente en el apoyo a personas con discapacidad o enfermedades que dificultan su capacidad de comunicarse verbalmente. Esta tecnología tiene un gran potencial para mejorar la calidad de vida de estas personas y facilitar su inclusión social. Por ello, se identifica que este trabajo contribuye principalmente a los siguientes Objetivos de Desarrollo Sostenible:

\begin{itemize}
	\item \textbf{ODS 10. Reducción de las desigualdades}: este trabajo contribuye a reducir las desigualdades en la sociedad, proporcionando soluciones tecnológicas que abordan las necesidades de personas con dificultades en la comunicación. Al mejorar la accesibilidad y la capacidad de comunicación, se eliminan barreras que tradicionalmente han limitado la participación plena de estas personas en la vida social. Además, al facilitar su integración en diversos contextos, se promueve una sociedad más inclusiva y equitativa donde todos los individuos pueden desarrollarse y participar activamente.
	\item \textbf{ODS 3. Salud y bienestar}: al mejorar las herramientas de comunicación, se favorece una mejor calidad de vida para las personas con problemas de comunicación, facilitando su integración y bienestar dentro de la comunidad. Además, estas herramientas pueden ser fundamentales en contextos de atención médica, donde la capacidad de comunicar necesidades y síntomas es crucial para recibir un cuidado adecuado. De este modo, se contribuye a garantizar una vida saludable y promover el bienestar de todas las personas.
	\item \textbf{ODS 4. Educación de calidad}: las tecnologías desarrolladas en este trabajo también tienen un impacto positivo en la educación inclusiva. Mejorando las herramientas de comunicación, se facilita el acceso a una educación de calidad para personas con discapacidad o enfermedades que afectan a sus capacidades de comunicación. Además, se fomenta un entorno educativo más inclusivo. Esto no solo contribuye a la equidad en la educación, sino que también ayuda a crear sociedades más diversas, tolerantes y comprensivas.
	\item \textbf{ODS 8. Trabajo decente y crecimiento económico}: las personas con discapacidad suelen enfrentar mayores obstáculos para conseguir empleos dignos. Al mejorar sus capacidades de comunicación, se incrementan sus oportunidades de obtener un empleo decente y de calidad, lo que contribuye a su autonomía económica y social.
\end{itemize}

Además, se reconoce un impacto menor, pero relevante, en los siguientes objetivos:

\begin{itemize}
	\item \textbf{ODS 1. Fin de la pobreza}: al mejorar la comunicación de personas con discapacidad, se incrementan sus oportunidades laborales y su capacidad para sostenerse económicamente. Esto no solo permite a estas personas alcanzar una mayor autonomía económica, sino que también reduce su vulnerabilidad a situaciones de pobreza, fomentando su inclusión en la vida económica y social de la comunidad.
	\item \textbf{ODS 16. Paz, justicia e instituciones sólidas}: fomentar la inclusión social de todas las personas, independientemente de sus capacidades, es esencial para construir sociedades más justas, equitativas y pacíficas. Al asegurar que todas las personas puedan participar en la vida social se fomenta el respeto por los derechos humanos y libertades fundamentales.
	\item \textbf{ODS 12. Producción y consumo responsables}: la implementación de tecnologías digitales de alta calidad puede reducir el uso de recursos materiales como el papel, promoviendo un consumo más sostenible.
\end{itemize}

En conclusión, este trabajo contribuye de manera significativa a varios de los ODS, particularmente en los ámbitos de inclusión social, salud, educación y empleo digno. Al mejorar las herramientas de comunicación para personas con dificultades, se promueve su participación activa en la sociedad, su acceso a servicios esenciales y su integración en entornos educativos y laborales, ayudando así a construir una sociedad más equitativa, justa y abierta a todas las personas.

\end{document}
